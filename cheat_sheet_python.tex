\documentclass[%
	11pt,
	a4paper,
	utf8,
	%twocolumn
		]{article}	

\usepackage{style_packages/podvoyskiy_article_extended}


\begin{document}
\title{Приемы программирования на языке Python}

\author{}

\date{}
\maketitle

\thispagestyle{fancy}

\tableofcontents

\section{Терминология}

Любой элемент данных, используемый в программе на Python, является \emph{объектом} \cite[\strbook{57}]{beazley:python-2010}.

Каждый объект имеет свою:
\begin{itemize}
	\item идентичность,
	
	\item тип (или класс),
	
	\item значение.
\end{itemize}

Например, когда в программе встречается интсрукция \verb|a = 42|, интерпретатор создает целочисленный объект со значением 42. Можно рассматривать идентичность объекта как указатель на область памяти, где находится объект, а индентификатор \texttt{a} -- как имя, которое ссылается на эту область памяти.

\emph{Тип объекта} сам по себе является \emph{объектом}, который называется \emph{классом объекта}. Все объекты в яызке Python могут быть отнесены к \emph{объектам первого класса} \cite[\strbook{61}]{beazley:python-2010}. Это означает, что все объекты, имеющие идентификатор, можно интерпретировать как \emph{данные}.

Тип \texttt{None} используется для представления пустых объектов (т.е. объектов, не имеющих значений). Этот объект возвращается функциями, которые не имеют явно возвращаемого значения. Объект \texttt{None} часто используется как значение по умолчанию для необязательных аргументов. Объкт \texttt{None} не имеет атрибутов и в логическом контексте оценивается как значение \texttt{False}



% Источники в "Газовой промышленности" нумеруются по мере упоминания 
\begin{thebibliography}{99}\addcontentsline{toc}{section}{Список литературы}
	\bibitem{koltzov-c-lang:2019}{ \emph{Кольцов Д.М.} Си на примерах. Практика, практика и только практика. -- СПб.: Наука и Техника, 2019. -- 288 с.}
	\bibitem{beazley:python-2010}{\emph{Бизли Д.} Python. Подробный справочник. -- СПб.: Символ-Плюс, 2010. -- 864 с.}
\end{thebibliography}

%\listoffigures\addcontentsline{toc}{section}{Список иллюстраций}

\lstlistoflistings\addcontentsline{toc}{section}{Список листингов}

\end{document}
